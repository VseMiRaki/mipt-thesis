\documentclass{mipt-thesis-bs}
% Следующие две строки нужны только для biblatex. Для inline-библиографии их следует убрать.
\usepackage{mipt-thesis-biblatex}
\addbibresource{main.bib}

\title{}
\author{Федорец Никита Сергеевич}
\supervisor{Дас Бисваруп}
%\referee{Петров Д.\,Е.}       % требуется только для mipt-thesis-ms
\groupnum{7910}
\faculty{{\bf Направление поготовки:} 01.03.02 Прикладная математика и информатика}
\department{{\bf Направленность подготовки:} Прикладная математика и компьютерные науки}

\begin{document}

% \frontmatter
\mainmatter
\titlepage
\thispagestyle{empty}

\chapter*{Аннотация} 
Здесь будет аннотация

\newpage

\tableofcontents 
% \titlecontents


% \chapter{Аннотация}


\chapter{Введение}
\section{start}
Здесь идет текст. Вот так выглядит ссылка на библиографию \cite{langmuir26}. 
Аналогично добавляются еще главы, внутри них можно объявлять секции с помощью \verb|\section|.
\section{end}
1234

\backmatter

\printbib
% Следующие строки необходимо раскомментировать, а предыдущую закомментировать, если используется inline-библиография.
%\begin{thebibliography}{99}
%    \bibitem{langmuir26}
%        H. Mott-Smith, I. Langmuir. ``The theory of collectors in gaseous discharges''. \emph{Phys. Rev.} \textbf{28} (1926)
%\end{thebibliography}

\chapter{Благодарности}

Благодарности идут тут.

\end{document}